
\documentclass[11pt]{article}
\usepackage{geometry} % see geometry.pdf on how to lay out the page. There's lots.
\usepackage{setspace}
\usepackage{boxedminipage}
\usepackage{color}
\usepackage{relsize}
\usepackage[greek,english]{babel}
\usepackage[colorinlistoftodos,shadow]{todonotes}
\usepackage{graphicx}
\usepackage{epstopdf}
\usepackage{url}
\usepackage{bm}
\usepackage{amsmath}
\usepackage{amsfonts}
\usepackage{amssymb}
\usepackage{array}
\usepackage{mathrsfs}
\usepackage[multiple]{footmisc}
\usepackage[para]{manyfoot}

\usepackage{tikz}
\usepackage{dcolumn}
\usepackage{caption}
\usepackage{subfig}


%\geometry{a4paper} % or letter or a5paper or ... etc
% \geometry{landscape} % rotated page geometry

% See the ``Article customise'' template for come common customisations

\title{MATLAB Codes: \\
	Small Open Economy Monetary Policy and Equilibrium Determinacy:
	New Lessons from a Model with Endogenous Monetary Policy Trade-Off }
\author{Jaime Alonso-Carrera and Timothy Kam}
\date{\today} % delete this line to display the current date

%%% BEGIN DOCUMENT
\begin{document}

\maketitle
\tableofcontents
\newpage

\section{Introduction}

The recursive competitive equilibrium is characterized by a three-equation forward-looking dynamical system:
\begin{equation}
\pi _{H,t}=\widehat{\beta }\mathbb{E}_{t}\left\{ \pi _{H,t}\right\} +\lambda
\left( \kappa _{1}\widetilde{x}_{t}+\kappa _{2}\widetilde{q}_{t}\right) ,
\label{Phillips}
\end{equation}%
\begin{equation}
\widetilde{x}_{t}=\varpi \mathbb{E}_{t}\left\{ \widetilde{x}_{t+1}\right\}
-\mu \left[ i_{t}-\mathbb{E}_{t}\left\{ \pi _{H,t+1}\right\} \right] +\chi 
\mathbb{E}_{t}\left\{ \widetilde{q}_{t+1}\right\} +\epsilon _{t},
\label{is equation}
\end{equation}%
and 
\begin{equation}
\widetilde{q}_{t}=\mathbb{E}_{t}\left\{ \widetilde{q}_{t+1}\right\} -\left(
1-\gamma \right) \left[ i_{t}-\mathbb{E}_{t}\left\{ \pi _{H,t+1}\right\} %
\right] +u_{t},  \label{exchange dynamics}
\end{equation}%
where%
\begin{equation*}
\lambda =\left[ \frac{\left( 1-\theta \right) \left( 1-\theta \beta \right) 
}{\theta }\right] \left[ \frac{\left( 1-v\right) \left( 1-\delta \right) }{%
1-v+\delta \varphi }\right] ,
\end{equation*}%
\begin{equation*}
\kappa _{1}=\frac{\sigma }{1-\gamma }+\varphi ,
\end{equation*}%
\begin{equation*}
\kappa _{2}=\underset{\text{Production effect}}{\underbrace{\frac{\delta
(1-v+\varphi )}{\left( 1-\gamma \right) \left( 1-v\right) (1-\delta )}}}-%
\underset{\text{demand effect}}{\underbrace{\frac{\sigma \eta \gamma \left(
2-\gamma \right) }{\left( 1-\gamma \right) ^{2}}+\frac{\gamma }{1-\gamma }}},
\end{equation*}%
\begin{equation*}
\varpi =\frac{\sigma }{\sigma -\phi },
\end{equation*}%
\begin{equation*}
\mu =\left( \frac{1-\gamma }{\sigma -\phi }\right) \left[ 1-\gamma +\frac{%
\eta \gamma \left( 2-\gamma \right) \left( \sigma -\phi \right) }{1-\gamma }%
\right] ,
\end{equation*}%
and%
\begin{equation*}
\chi =\frac{\eta \gamma \phi \left( 2-\gamma \right) }{\left( 1-\gamma
\right) \left( \sigma -\phi \right) }.
\end{equation*}

\section{Calibration}

Our baseline economy is defined by taking the same parameters as Llosa and
Tuesta (2008) and McCallum and Nelson (1999). Llosa and Tuesta (2008) uses
the same calibration as Gali and Monacelli (2005) with the exception of the
inverse of IES ($\sigma ),$ the inverse of Frisch labor supply elasticity ($%
\varphi ),$ and the elasticity of substitution between domestic and foreign
goods ($\eta ).$ However, the calibration of the former seems much
convenient for our analysis for two reasons: (i) because of comparison for
the stability analysis; and (ii) because is a much general calibration.
Furthermore, these parameters does not affect qualitatively to the results,
although they may have important quantitative effects. This is mainly true
in the case of $\sigma .$ In fact, we will perform some sensitivity analysis
in this parameter when it would be required.

The next table summarizes this parametrization:

\begin{center}
\begin{tabular}{|lllll|}
\hline
\textbf{Parameter} &  & \textbf{Value} &  & \textbf{Source} \\ 
\cline{1-1}\cline{3-3}\cline{5-5}
\textit{Preferences} &  &  &  &  \\ 
\multicolumn{1}{|c}{$\sigma $} &  & \multicolumn{1}{c}{$5$} &  & Llosa and
Tuesta (2008) \\ 
\multicolumn{1}{|c}{$\psi $} &  & \multicolumn{1}{c}{$-1$} &  & Gali and
Monacelli (2005) \\ 
\multicolumn{1}{|c}{$\varphi $} &  & \multicolumn{1}{c}{$0.47$} &  & Llosa
and Tuesta (2008) \\ 
\multicolumn{1}{|c}{$\phi $} &  & \multicolumn{1}{c}{$10^{-6}$} &  &  \\ 
\multicolumn{1}{|c}{$\vartheta $} &  & \multicolumn{1}{c}{$0$} &  &  \\ 
\multicolumn{1}{|c}{$\beta $} &  & \multicolumn{1}{c}{$0.99$} &  & Gali and
Monacelli (2005) \\ 
\textit{Composition demand} &  & \multicolumn{1}{c}{} &  &  \\ 
\multicolumn{1}{|c}{$\eta $} &  & \multicolumn{1}{c}{$1.5$} &  & Llosa and
Tuesta (2008) \\ 
\multicolumn{1}{|c}{$\gamma $} &  & \multicolumn{1}{c}{$0.4$} &  & Gali and
Monacelli (2005) \\ 
\multicolumn{1}{|c}{$\varepsilon $} &  & \multicolumn{1}{c}{$6$} &  & Gali
and Monacelli (2005) \\ 
\multicolumn{1}{|c}{$sc$} &  & \multicolumn{1}{c}{$0.75$} &  & Cooley and
Prescott (1995) \\ 
&  & \multicolumn{1}{c}{$0.89$} &  & McCallum and Nelson (1999) \\ 
\textit{Production} &  & \multicolumn{1}{c}{} &  &  \\ 
\multicolumn{1}{|c}{$\theta $} &  & \multicolumn{1}{c}{$0.75$} &  & Gali and
Monacelli (2005) \\ 
\multicolumn{1}{|c}{$v$} &  & \multicolumn{1}{c}{$-2$} &  & McCallum and
Nelson (1999) \\ 
\multicolumn{1}{|c}{$\delta $} &  & \multicolumn{1}{c}{$0.144$} &  & 
McCallum and Nelson (1999) \\ \hline
\end{tabular}
\end{center}

The parameter $v$ is chosen by McCallum and Nelson (2005) to avoid a
excessive variability of the output under flexible prices with respect to
real exchange rate. In our model this variability is given by $\Omega _{2}.$
As in the aforementioned paper, this volatility is small for values of $v$
smaller than $-2.$

The previous would be our baseline economy. However, we will compare in our
numerical discussion and analysis between four different scenarios:

\begin{enumerate}
\item Our baseline economy: the open economy with incomplete markets.

\item Our extreme example followed to derive the optimal policy: $\phi =0,$ $%
\alpha =1$ (i.e., $\delta =0)$.

\item The open economy with complete markets: $\kappa _{2}=0,$ $\kappa
_{1}=\kappa _{1}^{c}$ and $\mu =\mu ^{c}.$ Moreover, the other characterization of this economy is given by the complete risk sharing condition, in place of the UIP condition.

\item The close economy: $\gamma =0$ and $\alpha =1$ (i.e., $\delta =0).$
\end{enumerate}

Afterwards we may also consider the intervals of the parameters that
determines the stability of the equilibrium under different monetary
policies. These parameters could be: (i) those determining the openness via
demand, i.e., $\eta $ and $\gamma ;$ (ii) those determining the openness via
production, i.e., $\alpha $ and $v;$ (iii) those determining the frictions,
i.e., $\varepsilon $ and $\theta ;$ and (iv) the inverse of the
intertemporal elasticity of substitution $\sigma$.

\section{Numerical Experiments}

We will consider three policy rules:

\begin{enumerate}
\item The \textit{Domestic Inflation Taylor Rule} (DITR).- Where the
domestic monetary authority adjust the domestic interest rate to both
domestic inflation and the domestic output gap:%
\begin{equation}
i_{t}=\phi _{\pi }\pi _{H,t}+\phi _{x}\widetilde{x}_{t}  \label{DITR}
\end{equation}%
where $\phi _{\pi }$ and $\phi _{x}$ are exogenous non-negative reaction
parameters.

\item The\textit{\ CPI Inflation Taylor Rule} (CPITR).- Where the monetary
authority adjust the domestic interest rate to both CPI inflation and the
domestic gap:%
\begin{equation}
i_{t}=\phi _{\pi }\pi _{t}+\phi _{x}\widetilde{x}_{t},  \label{CPITR}
\end{equation}%
where $\phi _{\pi }$ and $\phi _{x}$ are again exogenous non-negative
reaction parameters.

\item The \textit{Managed Exchange Rate Taylor Rule} (MERTR).- Where the
domestic monetary authority adjust the domestic interest rate to CPI
inflation, the domestic output gap and the change in the nominal exchange
rate:%
\begin{equation}
i_{t}=\phi _{\pi }\pi _{t}+\phi _{x}\widetilde{x}_{t}+\phi
_{s}\bigtriangleup s_{t}  \label{MERTR}
\end{equation}%
where $\bigtriangleup s_{t}$ is the change in the nominal exchange rate, and 
$\phi _{\pi },$ $\phi _{x}$ and $\phi _{s}$ are again exogenous non-negative
reaction parameters.

\item The \textit{Optimal Policy Rule} (OPR).- We have obtained that the
optimal condition of the minimization problem of the Loss Function is%
\begin{equation}
a\widetilde{q}_{t}+b\widetilde{x}_{t}+c\pi _{H,t}+\widetilde{z}_{t}=0,
\label{OPR}
\end{equation}%
where $a,$ $b$ and $c$ are reduced parameters that depend on the fundamental
parameters, and $\widetilde{z}_{t}$ is an exogenous stochastic variable.
This will be a monetary rule of the form%
\begin{equation*}
i_{t}=\widehat{\phi }_{\pi }E\left\{ \pi _{H,t+1}\right\} +\widehat{\phi }%
_{x}E\left\{ \widetilde{x}_{t+1}\right\} +\widehat{\phi }_{q}E\left\{ 
\widetilde{q}_{t+1}\right\} ,
\end{equation*}%
where $\widehat{\phi }_{\pi },$ $\widehat{\phi }_{x}$ and $\widehat{\phi }%
_{q}$ are now \textbf{endogenous} reaction parameters (i.e., they depend on
the fundamental parameters of our model). For that reason we first study the
stability of the following ad-hoc rule\ :%
\begin{equation}
i_{t}=\phi _{\pi }E\left\{ \pi _{H,t+1}\right\} +\phi _{x}E\left\{ 
\widetilde{x}_{t+1}\right\} +\phi _{q}E\left\{ \widetilde{q}_{t+1}\right\} ,
\label{FB-ERTR}
\end{equation}%
where $\phi _{\pi },$ $\phi _{x}$ and $\phi _{q}$ are again exogenous
non-negative reaction parameters. After that, we should check if the
parameters of the optimal rule $\widehat{\phi }_{\pi },$ $\widehat{\phi }_{x}
$ and $\widehat{\phi }_{q}$ belong to the set of parameters $\left\{ \phi
_{\pi },\phi _{x},\phi _{q}\right\} $ that report determinacy in the previous%
\textit{\ Forecast-Based Managed Exchange Rate Taylor Rule (FB-MERTR)}.
\end{enumerate}

\section{\texttt{MATLAB} Scripts}

\noindent\textbf{Step 1.} The relevant \texttt{MATLAB} scripts are stored in the directory \texttt{simulations/} of the accompanying files to the paper. Below we list a short description of each experiment/script:
\begin{itemize}
\item Domestic Inflation Targeting Rule (DITR) and Forecast-based DITR (FB-DITR):
	\begin{itemize}
	\item \texttt{ditr\_{close}.m}: For the closed economy special case of our model.
	\item \texttt{ditr\_complete.m}: For the Gali-Monacelli complete markets open economy special case of our model.
	\item \texttt{ditr.m}: For our model with incomplete markets.
	\end{itemize}
	
\item CPI Inflation Targeting Rule (CPITR) and Forecast-based CPITR (FB-CPITR):
	\begin{itemize}
	\item \texttt{cpitr\_{close}.m}: For the closed economy special case of our model.
	\item \texttt{mertrCPI\_complete.m}: Set \texttt{POLICY = 0} for this rule in the script. For the Gali-Monacelli complete markets open economy special case of our model.
	\item \texttt{mertrCPI.m}: Set \texttt{POLICY = 0} for this rule in the script. For our model with incomplete markets.
	\end{itemize}
	
\item Managed Exchange Rate Taylor Rule (MERTR) and Forecast-based MERTR (FB-MERTR):
	\begin{itemize}
	\item \texttt{mertrCPI\_complete.m}: Set \texttt{POLICY = 1} for this rule in the script.  For the Gali-Monacelli complete markets open economy special case of our model.
	\item \texttt{mertrCPI.m}: Set \texttt{POLICY = 1} for this rule in the script. For our model with incomplete markets.
	\end{itemize}	
	
\item Optimal Time-Consistent Managed Exchange Rate Taylor Rule (MERTR) and Forecast-based MERTR (FB-MERTR):
	\begin{itemize}
	\item \texttt{mertrCPI\_complete.m}: Set \texttt{POLICY = 1} for the FB-MERTR rule in the script.  For the Gali-Monacelli complete markets open economy special case of our model.
	\item \texttt{mertrCPI.m}: Set \texttt{POLICY = 1} for the FB-MERTR rule in the script. For our model with incomplete markets.
	\item \texttt{mertr\_optimal}: Compute stability regions under a set of FB-MERTR where the $\phi_{q}$ parameter is changed; for the incomplete markets model. Also computes the point corresponding to the baseline calibrated model's optimal time-consistent policy rule, expressed as a point in the FB-MERTR family of rules.
	\end{itemize}	
\end{itemize}

\noindent\textbf{Step 2.} A summary script is found in $\texttt{patchdraw.m}$. This collects all the results above and performs a convex hull approximation of relevant regions of $(\phi_{\pi}, \phi_{x})$ that induces stable rational expectations equilibrium for each fixed policy behavior. This script then graphs and saves the output as ``patch'' diagrams. The figures are named and saved automatically according to the name of the family of policy rules considered (e.g. \texttt{ditr.eps} and \texttt{ditr.fig} for the DITR family of rules) in a sub-directory called \texttt{simulations/\_{figures}/}.

\subsection{User instructions}

\begin{itemize}
\item Perform all the possible combinations of the script executions in Step1 above.
\item Then run Step 2.
\end{itemize}

\end{document}