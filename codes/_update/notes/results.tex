
\documentclass{article}
%%%%%%%%%%%%%%%%%%%%%%%%%%%%%%%%%%%%%%%%%%%%%%%%%%%%%%%%%%%%%%%%%%%%%%%%%%%%%%%%%%%%%%%%%%%%%%%%%%%%%%%%%%%%%%%%%%%%%%%%%%%%%%%%%%%%%%%%%%%%%%%%%%%%%%%%%%%%%%%%%%%%%%%%%%%%%%%%%%%%%%%%%%%%%%%%%%%%%%%%%%%%%%%%%%%%%%%%%%%%%%%%%%%%%%%%%%%%%%%%%%%%%%%%%%%%
\usepackage{amsmath}
\usepackage{amsfonts}

\setcounter{MaxMatrixCols}{10}
%TCIDATA{OutputFilter=LATEX.DLL}
%TCIDATA{Version=5.50.0.2953}
%TCIDATA{<META NAME="SaveForMode" CONTENT="1">}
%TCIDATA{BibliographyScheme=Manual}
%TCIDATA{Created=Friday, May 14, 2010 12:14:55}
%TCIDATA{LastRevised=Monday, July 12, 2010 12:33:17}
%TCIDATA{<META NAME="GraphicsSave" CONTENT="32">}
%TCIDATA{<META NAME="DocumentShell" CONTENT="Standard LaTeX\Blank - Standard LaTeX Article">}
%TCIDATA{Language=American English}
%TCIDATA{CSTFile=40 LaTeX article.cst}
%TCIDATA{ComputeDefs=
%$\kappa _{1}^{c}=\left[ \frac{1}{{}}\right] \left( \frac{\sigma }{1-\gamma }%
%\right) +\varphi $
%}


\special{papersize=29.7cm,21cm}
\textheight 22cm
\textwidth 14,5cm
\oddsidemargin 0.75cm
\headsep 0cm
\input tcilatex
\begin{document}

\title{On a tractable Small Open Economy Model with Endogenous
Monetary-Policy Trade-offs\vspace{1.5cm}\\
Numerical analysis\vspace{1.5cm}}
\author{Jaime Alonso-Carrera and Timothy Kam}
\maketitle

\newpage

\section{Approximation of the competitive equilibrium}

The competitive equilibrium is now represented by three equations:%
\begin{equation}
\pi _{H,t}=\widehat{\beta }\mathbb{E}_{t}\left\{ \pi _{H,t}\right\} +\lambda
\left( \kappa _{1}\widetilde{x}_{t}+\kappa _{2}\widetilde{q}_{t}\right) ,
\label{Phillips}
\end{equation}%
\begin{equation}
\widetilde{x}_{t}=\varpi \mathbb{E}_{t}\left\{ \widetilde{x}_{t+1}\right\}
-\mu \left[ i_{t}-\mathbb{E}_{t}\left\{ \pi _{H,t+1}\right\} \right] +\chi 
\mathbb{E}_{t}\left\{ \widetilde{q}_{t+1}\right\} +\epsilon _{t},
\label{is equation}
\end{equation}%
and 
\begin{equation}
\widetilde{q}_{t}=\mathbb{E}_{t}\left\{ \widetilde{q}_{t+1}\right\} -\left(
1-\gamma \right) \left[ i_{t}-\mathbb{E}_{t}\left\{ \pi _{H,t+1}\right\} %
\right] +u_{t},  \label{exchange dynamics}
\end{equation}%
where%
\begin{equation*}
\lambda =\left[ \frac{\left( 1-\theta \right) \left( 1-\theta \beta \right) 
}{\theta }\right] \left[ \frac{\left( 1-v\right) \left( 1-\delta \right) }{%
1-v+\delta \varphi }\right] ,
\end{equation*}%
\begin{equation*}
\kappa _{1}=\frac{\sigma }{1-\gamma }+\varphi ,
\end{equation*}%
\begin{equation*}
\kappa _{2}=\underset{\text{Production effect}}{\underbrace{\frac{\delta
(1-v+\varphi )}{\left( 1-\gamma \right) \left( 1-v\right) (1-\delta )}}}-%
\underset{\text{demand effect}}{\underbrace{\frac{\sigma \eta \gamma \left(
2-\gamma \right) }{\left( 1-\gamma \right) ^{2}}+\frac{\gamma }{1-\gamma }}},
\end{equation*}%
\begin{equation*}
\varpi =\frac{\sigma }{\sigma -\phi },
\end{equation*}%
\begin{equation*}
\mu =\left( \frac{1-\gamma }{\sigma -\phi }\right) \left[ 1-\gamma +\frac{%
\eta \gamma \left( 2-\gamma \right) \left( \sigma -\phi \right) }{1-\gamma }%
\right] ,
\end{equation*}%
and%
\begin{equation*}
\chi =\frac{\eta \gamma \phi \left( 2-\gamma \right) }{\left( 1-\gamma
\right) \left( \sigma -\phi \right) }.
\end{equation*}

What is new in our model?:

\begin{enumerate}
\item \textbf{Exchange rate now matters}. First, the current value of this
rate appears in the New Keynesian Phillips curve. In addition, the expected
value of the exchange rate also appears in the IS equation. However, this
last effect should be discarded because depends on the assumption of
endogenous discounting. Since we assume that the value of $\phi $ is very
close to zero, this effect tends to disappear. Focusing on the effect of
exchange rate on the inflation, this is given by $\lambda \kappa _{2}.$
Observe that $\kappa _{2}$ can be decomposed into two effects: \textit{%
production effect} and \textit{demand effect}. Let us explain them:

\begin{itemize}
\item[(a)] \underline{Production effect}.- A depreciation of the exchange
rate increases the prices of the imported inputs, which increases the
marginal cost of producing and then the domestic inflation.

\item[(b)] \underline{Demand effect}.- This effect has two components. A
deprecation of the exchange rate provokes a substitution of home consumption
goods by foreign goods, which increases the marginal cost of producing and
then the domestic inflation. In addition the depreciation of exchange rate
has also an income effect, the depreciation forces the consumers to reduce
the consumption of domestic goods if they want to maintain the level of
consumption of foreign goods. This reduction of the demand of domestic
goods, reduces the marginal cost and then the domestic inflation rate. A NEW
TASK: We have to decompose the second part of $\kappa _{2}$ into the two
components of this demand effect.
\end{itemize}

\item In our economy \textbf{openness depends on two parameters}: $\gamma ,$
which determines the exterior dependence of demand; and $\delta ,$ which
determines the exterior dependence of production. This arises some important
issues:

\begin{itemize}
\item[(a)] The effects of $\gamma $ are different to the original model of
Gali and Monacelli (2005). As in this paper, in our economy $\gamma $ also
affects the response of the output gap to the real interest rate ($\mu )$
and the response of inflation to the current output gap ($\kappa _{1}).$
However, this parameter $\gamma $ also affect in our model to the response
of inflation to the current exchange rate ($\kappa _{2}).$

\item[(b)] In contrast with Gali and Monacelli (2005), and as McCallum and
Nelson (1999), openness also matters in the production side. The parameter $%
\delta $ first affects the response of inflation to the current output gap ($%
\lambda ),$ and to the current exchange rate ($\lambda $ and $\kappa _{2}).$
However, in contrast with the demand openness $\gamma ,$ this channel of
openness does not affect the response of output gap to the real interest
rate ($\mu ).$

\item[(c)] Regarding the effect of openness on the response of domestic
inflation to output gap, note that the effect is ambiguous in our model
because openness in demand ($\gamma )$ positively affects $\kappa _{1},$
whereas openness in production ($\delta )$ negatively affects $\lambda .$

\item[(d)] These channels of openness effect should be considered to analyze
the policy trade-off and the stability of the competitive equilibrium.
\end{itemize}

\item \textbf{What is the role of incomplete markets in the equilibrium
relations?} To answer this question, we should derive the complete version
of our model, which is a little be different than the one in Gali and
Monacelli (2005) because our assumption on the openness in the production
side (i.e., $\delta \neq 0).$ In complete markets the international risk
sharing imposes that $q_{t}=\sigma \left( c_{t}+c_{t}^{\ast }\right) ,$
Using this relationship we obtain that the expression (30) in the paper
transforms into%
\begin{equation*}
y_{t}=\left[ 1-\gamma +\frac{\gamma \sigma \eta \left( 2-\gamma \right) }{%
1-\gamma }\right] c_{t}+\left[ \gamma +\frac{\gamma \sigma \eta \left(
2-\gamma \right) }{1-\gamma }\right] c_{t}^{\ast }.
\end{equation*}%
This implies that output is a linear function of exchange rate, so that they
are perfect correlated. In particular, we obtain that%
\begin{equation}
q_{t}=\left[ \underset{\tau }{\underbrace{\frac{\sigma \left( 1-\gamma
\right) }{\left( 1-\gamma \right) ^{2}+\eta \sigma \gamma \left( 2-\gamma
\right) }}}\right] y_{t.}  \label{complete relation}
\end{equation}%
By using this new expression, we obtain that the differences between the
incomplete market model and the complete market model is in three points:

\begin{itemize}
\item[(a)] In the complete version the exchange rate does not affect the
inflation, i.e., $\kappa _{2}^{c}=0.\footnote{%
We will use the super-script $c$ to denote the parameters that are different
in the complete market version of the model.}$ This implies that the
complete version is an isomorphism of the close model. This is true because
even when the expected exchange rate affects the output gap, this effect is
negligible because $\phi $ is almost zero, and so also $\chi .$
Interestingly, remember that the fact that $\kappa _{2}\neq 0$ implies that
incompleteness introduces endogenously a cost push in the New Keynesian
Phillips curve.

\item[(b)] Incompleteness also affects the response of inflation to the
output gap. Effectively, the parameter $\kappa _{1}$ is different in the
complete version. In particular, it would be%
\begin{equation*}
\kappa _{1}^{c}=\kappa _{1}+\tau \kappa _{2}=\left\{ \frac{1-v+\delta
\varphi }{\left( 1-\delta \right) \left( 1-v\right) \left[ 1-\gamma +\frac{%
\gamma \sigma \eta \left( 2-\gamma \right) }{1-\gamma }\right] }\right\}
\left( \frac{\sigma }{1-\gamma }\right) +\varphi .
\end{equation*}%
The difference between $\kappa _{1}$ (incomplete markets) and $\kappa
_{1}^{c}$ (complete markets) is the first expression in the right-hand side
of $\kappa _{1}^{c}.$ This difference has an ambiguous sign because a priory
is difficult to sign $1-v+\delta \varphi .$ However, we will consider
negative values of $v.$ Hence, this first expression in the right-hand side
of $\kappa _{1}^{c}$ will be positive. However, we do not yet know whether
this expression is larger or smaller than one. In the next section we will
quantify this difference.

\item[(c)] Finally, incompleteness also affects the response of output gap
to the real interest rate given by $\mu .$ In the complete market version
this parameter would be%
\begin{equation*}
\mu ^{c}=\left[ \frac{1-\gamma }{\sigma -\phi \left( 1-\gamma \right) }%
\right] \left[ 1-\gamma +\frac{\eta \gamma \sigma \left( 2-\gamma \right) }{%
1-\gamma }\right]
\end{equation*}%
However, provided that $\phi $ takes values very close to zero, then $\mu
=\mu ^{c}.$ Therefore, the effect of incompleteness in the response of
output gap to real interest rate is negligible.
\end{itemize}

Therefore, the relevant difference introduced by assuming incomplete markets
is in the New Keynesian Phillips curve. WE SHOULD FIND INTUITIONS FOR THESE
DIFFERENCES.

\item \textbf{A caveat}.- Our model with complete markets is not equal to
the model in Gali and Monacelli (2005). One obvious reason is that we
incorporate the openness in production, which affects the response of
inflation rate to output gap $\lambda \kappa _{1}^{c}.$ However, this
discrepancy with Gali and Monacelli (2005) also arise in the response of
output gap to real interest rate $\mu ^{c}$ even when this parameter does
not depend on openness in production $\delta .$ In particular, this
parameter $\mu ^{c}$ in Gali and Monacelli (2005) framework with endogenous
discounting would be 
\begin{equation*}
\mu ^{gm}=\left[ \frac{1-\gamma }{\sigma -\phi \left( 1-\gamma \right) }%
\right] \left[ 1-\gamma +\frac{\sigma \gamma \left[ \widehat{\gamma }+\eta
\left( 1-\gamma \right) \right] }{1-\gamma }\right] .
\end{equation*}%
The difference is the parameter $\widehat{\gamma }$ that in our model is
equal to $\eta .$ What is the economic meaning of this parameter $\widehat{%
\gamma }?$ Gali and Monacelli (2005) consider that each imported good $j$ is
a CES aggregator of the quantity purchased in each foreign country $i,$
where $\widehat{\gamma }$ is the elasticity of substitution across
countries, i.e., 
\begin{equation*}
C_{F}(j)=\left[ \int_{0}^{1}\left[ C_{i}(j)\right] ^{\frac{\widehat{\gamma }%
-1}{\widehat{\gamma }}}\right] ^{\frac{\widehat{\gamma }}{\widehat{\gamma }-1%
}}.
\end{equation*}%
We follow Clarida et al. (2001) in treating the rest of the world as a
single, non-small country. This implies to assume that $\widehat{\gamma }%
=\eta .$

QUESTION: what does this mean in terms of substitutability between
countries? Therefore, our model with complete markets coincides with Gali
and Monacelli (2005) under they calibration: $\widehat{\gamma }=\eta =1.$
However, both models do not exactly coincide under the calibration in Llosa
and Tuesta (2006): $\widehat{\gamma }=1$ and $\eta =1.5.$ We will consider
in our calibration that $\eta =1.5,$ so that we will be assuming that $%
\widehat{\gamma }=1.5.$ This point should be clear. In any case, we think
that this point is negligible for the numerical results.

TO BE DISCUSSED.- Should we change our model to be compared with Gali and
Monacelli (2005) and Llosa and Tuesta (2006). I think that this change is
not necessary because the negligible quantitative consequences of this
deviation. We can say that we consider the Clarida et al. (2001) version of
the demand side to consider the rest of the world as a non-small country.
\end{enumerate}

\textbf{Note.-} The second part of $\lambda $ and the first pat of $\kappa
_{2}$ is wrong in the paper. The true expressions are those given above.

\section{Calibration}

Our baseline economy is defined by taking the same parameters as Llosa and
Tuesta (2008) and McCallum and Nelson (1999). Llosa and Tuesta (2008) uses
the same calibration as Gali and Monacelli (2005) with the exception of the
inverse of IES ($\sigma ),$ the inverse of Frisch labor supply elasticity ($%
\varphi ),$ and the elasticity of substitution between domestic and foreign
goods ($\eta ).$ However, the calibration of the former seems much
convenient for our analysis for two reasons: (i) because of comparison for
the stability analysis; and (ii) because is a much general calibration.
Furthermore, these parameters does not affect qualitatively to the results,
although they may have important quantitative effects. This is mainly true
in the case of $\sigma .$ In fact, we will perform some sensitivity analysis
in this parameter when it would be required.

The next table summarizes this parametrization:

\begin{center}
\begin{tabular}{|lllll|}
\hline
\textbf{Parameter} &  & \textbf{Value} &  & \textbf{Source} \\ 
\cline{1-1}\cline{3-3}\cline{5-5}
\textit{Preferences} &  &  &  &  \\ 
\multicolumn{1}{|c}{$\sigma $} &  & \multicolumn{1}{c}{$5$} &  & Llosa and
Tuesta (2008) \\ 
\multicolumn{1}{|c}{$\psi $} &  & \multicolumn{1}{c}{$-1$} &  & Gali and
Monacelli (2005) \\ 
\multicolumn{1}{|c}{$\varphi $} &  & \multicolumn{1}{c}{$0.47$} &  & Llosa
and Tuesta (2008) \\ 
\multicolumn{1}{|c}{$\phi $} &  & \multicolumn{1}{c}{$10^{-6}$} &  &  \\ 
\multicolumn{1}{|c}{$\vartheta $} &  & \multicolumn{1}{c}{$0$} &  &  \\ 
\multicolumn{1}{|c}{$\beta $} &  & \multicolumn{1}{c}{$0.99$} &  & Gali and
Monacelli (2005) \\ 
\textit{Composition demand} &  & \multicolumn{1}{c}{} &  &  \\ 
\multicolumn{1}{|c}{$\eta $} &  & \multicolumn{1}{c}{$1.5$} &  & Llosa and
Tuesta (2008) \\ 
\multicolumn{1}{|c}{$\gamma $} &  & \multicolumn{1}{c}{$0.4$} &  & Gali and
Monacelli (2005) \\ 
\multicolumn{1}{|c}{$\varepsilon $} &  & \multicolumn{1}{c}{$6$} &  & Gali
and Monacelli (2005) \\ 
\multicolumn{1}{|c}{$sc$} &  & \multicolumn{1}{c}{$0.75$} &  & Cooley and
Prescott (1995) \\ 
&  & \multicolumn{1}{c}{$0.89$} &  & McCallum and Nelson (1999) \\ 
\textit{Production} &  & \multicolumn{1}{c}{} &  &  \\ 
\multicolumn{1}{|c}{$\theta $} &  & \multicolumn{1}{c}{$0.75$} &  & Gali and
Monacelli (2005) \\ 
\multicolumn{1}{|c}{$v$} &  & \multicolumn{1}{c}{$-2$} &  & McCallum and
Nelson (1999) \\ 
\multicolumn{1}{|c}{$\delta $} &  & \multicolumn{1}{c}{$0.144$} &  & 
McCallum and Nelson (1999) \\ \hline
\end{tabular}
\end{center}

The parameter $v$ is chosen by McCallum and Nelson (2005) to avoid a
excessive variability of the output under flexible prices with respect to
real exchange rate. In our model this variability is given by $\Omega _{2}.$
As in the aforementioned paper, this volatility is small for values of $v$
smaller than $-2.$

The previous would be our baseline economy. However, we will compare in our
numerical discussion and analysis between four different scenarios:

\begin{enumerate}
\item Our baseline economy: the open economy with incomplete markets.

\item Our extreme example followed to derive the optimal policy: $\phi =0,$ $%
\alpha =1$ (i.e., $\delta =0)$.

\item The open economy with complete markets: $\kappa _{2}=0,$ $\kappa
_{1}=\kappa _{1}^{c}$ and $\mu =\mu ^{c}.$ WE SHOULD COMPUTE THE EQUILIBRIUM
FROM THE BEGINNING USING THE FACT THAT PERFECT SHARING IMPLIES $%
c_{t}=q_{t}/\sigma ,$ WHICH SHOULD BE SUBSTITUTED INTO $(30).$

\item The close economy: $\gamma =0$ and $\alpha =1$ (i.e., $\delta =0).$
\end{enumerate}

Afterwards we may also consider the intervals of the parameters that
determines the stability of the equilibrium under different monetary
policies. These parameters could be: (i) those determining the openness via
demand, i.e., $\eta $ and $\gamma ;$ (ii) those determining the openness via
production, i.e., $\alpha $ and $v;$ (iii) those determining the frictions,
i.e., $\varepsilon $ and $\theta ;$ and (iv) the inverse of the
intertemporal elasticity of substitution $\sigma .$

\section{Numerical values of equilibrium relations}

By using our baseline calibration we can already sign the reduced parameters
determining the equilibrium relations between variables in the approximation
of the competitive equilibrium (\ref{Phillips}), (\ref{is equation}) and (%
\ref{exchange dynamics}). In particular we obtain

\begin{center}
\begin{equation*}
\begin{tabular}{|ccc|}
\hline
\multicolumn{3}{|c|}{Baseline Economy} \\ \hline\hline
Parameter &  & Value \\ \cline{1-1}\cline{3-3}
$\lambda $ &  & $0.0719$ \\ 
$\kappa _{1}$ &  & $8.8033$ \\ 
$\kappa _{2}$ &  & $-12.3424$ \\ 
$\varpi $ &  & $1.0000$ \\ 
$\mu $ &  & $1.0320$ \\ 
$\chi $ &  & $3.2\times 10^{-7}$ \\ \hline
\end{tabular}%
\end{equation*}
\end{center}

The unique unclear equilibrium relation is the dependence of the output gap
from the real exchange rate in the New Keynesian Phillips curve, given by $%
\lambda \kappa _{2}.$ From Mundell-Flemming one should expect that without
imported inputs this value should be negative. We show that in our baseline
economy, this value is also negative with import inputs. In fact, we obtain
that $\kappa _{2}$ is positive if and only if $v\in \left( -46.086,1\right)
. $ (Remember that $v\leq 1,$ and $v=0$ is the Cobb-Douglas production
function). Therefore, the aforementioned income effects of the exchange rate
variations dominates.

At this point, it is convenient quantify the differences between the
incomplete market version and the complete market version in terms of
equilibrium relations. Remember that this differences are in the relations $%
\lambda \kappa _{1},$ $\lambda \kappa _{2}$ and $\mu .$ The next table
reports this differences with and without openness in production:

\begin{equation*}
\begin{tabular}{ccc|cc|c}
& \multicolumn{2}{c|}{Incomplete} & \multicolumn{2}{|c|}{Complete} & Close
\\ \cline{2-6}
& $\delta =0.114$ & $\delta =0$ & $\delta =0.114$ & $\delta =0$ & $\delta
=\gamma =0$ \\ \hline\hline
$\lambda \kappa _{1}$ & $0.6325$ & $0.7556$ & $0.1169$ & $0.1235$ & $0.4695$
\\ 
$\lambda \kappa _{2}$ & $-0.8868$ & $-1.0872$ & $0$ & $0$ & $0$ \\ 
$\mu $ & $1.0320$ & $1.0320$ & $1.0320$ & $1.0320$ & $0.2$ \\ \hline
\end{tabular}%
\end{equation*}%
From this table we conclude the following:

\begin{enumerate}
\item The positive response of the inflation rate to the output gap, given
by $\lambda \kappa _{1},$ is much larger with incomplete markets. FIND AN
EXPLANATION.

\item This response of the inflation rate to the output gap, given by $%
\lambda \kappa _{1},$ is decreasing with $\delta $ (openness in production).

\item The response of the inflation rate to the output gap, given by $%
\lambda \kappa _{1},$ in the close economy is between the value in the
incomplete market version and the complete market version.

\item The openness in production, given by $\delta ,$ reduces the negative
effect of the depreciation of the exchange rate on the inflation rate. This
is obvious because with $\delta =0,$ the production effect determining $%
\kappa _{2}$ disappears.

\item Given that $\phi $ is very close to zero, the response of the output
gap to the interest rate, given by $\mu ,$ is the same in the two versions
of open economies.

\item The response of the output gap to the interest rate, given by $\mu ,$
is much smaller in the close economy.
\end{enumerate}

Using the value of $\sigma =1$ in Gali and Monacelli we obtain:

\begin{equation*}
\begin{tabular}{|ccc|}
\hline
\multicolumn{3}{|c|}{$\sigma =1$} \\ \hline\hline
Parameter &  & Value \\ \cline{1-1}\cline{3-3}
$\lambda $ &  & $0.0719$ \\ 
$\kappa _{1}$ &  & $2.1367$ \\ 
$\kappa _{2}$ &  & $-1.6757$ \\ 
$\varpi $ &  & $1.0000$ \\ 
$\mu $ &  & $1.0320$ \\ 
$\chi $ &  & $1.6\times 10^{-6}$ \\ \hline
\end{tabular}%
\end{equation*}

We observe that the important change with respect to the baseline economy is 
$\kappa _{2},$ which determines the effect of real exchange rate in the
inflation. There is also an important change in $\kappa _{1}$; which
determines the response of inflation to output gap. However these changes
only affect the length but not the sign. We should mention this point. In
general, the change in the calibration with respect to the original in Gali
and Monacelli implies important variations in the equilibrium relations.
Although the main source of this change is the new value of $\sigma $ (i.e.,
the inverse of the intertemporal elasticity of substitution). The $\eta $
(i.e., the elasticity of substitution between domestic and foreign goods)
has important impacts on $\kappa _{2}$ but much smaller than $\sigma ,$ and $%
\varphi $ (i.e., the inverse of the Frisch labor supply elasticity) has a
very small impact on $\kappa _{2}.$

\section{Policy rules}

We will consider three policy rules:

\begin{enumerate}
\item The \textit{Domestic Inflation Taylor Rule} (DITR).- Where the
domestic monetary authority adjust the domestic interest rate to both
domestic inflation and the domestic output gap:%
\begin{equation}
i_{t}=\phi _{\pi }\pi _{H,t}+\phi _{x}\widetilde{x}_{t}  \label{DITR}
\end{equation}%
where $\phi _{\pi }$ and $\phi _{x}$ are exogenous non-negative reaction
parameters.

\item The\textit{\ CPI Inflation Taylor Rule} (CPITR).- Where the monetary
authority adjust the domestic interest rate to both CPI inflation and the
domestic gap:%
\begin{equation}
i_{t}=\phi _{\pi }\pi _{t}+\phi _{x}\widetilde{x}_{t},  \label{CPITR}
\end{equation}%
where $\phi _{\pi }$ and $\phi _{x}$ are again exogenous non-negative
reaction parameters.

\item The \textit{Managed Exchange Rate Taylor Rule} (MERTR).- Where the
domestic monetary authority adjust the domestic interest rate to CPI
inflation, the domestic output gap and the change in the nominal exchange
rate:%
\begin{equation}
i_{t}=\phi _{\pi }\pi _{t}+\phi _{x}\widetilde{x}_{t}+\phi
_{s}\bigtriangleup s_{t}  \label{MERTR}
\end{equation}%
where $\bigtriangleup s_{t}$ is the change in the nominal exchange rate, and 
$\phi _{\pi },$ $\phi _{x}$ and $\phi _{s}$ are again exogenous non-negative
reaction parameters.

\item The \textit{Optimal Policy Rule} (OPR).- We have obtained that the
optimal condition of the minimization problem of the Loss Function is%
\begin{equation}
a\widetilde{q}_{t}+b\widetilde{x}_{t}+c\pi _{H,t}+\widetilde{z}_{t}=0,
\label{OPR}
\end{equation}%
where $a,$ $b$ and $c$ are reduced parameters that depend on the fundamental
parameters, and $\widetilde{z}_{t}$ is an exogenous stochastic variable.
This will be a monetary rule of the form%
\begin{equation*}
i_{t}=\widehat{\phi }_{\pi }E\left\{ \pi _{H,t+1}\right\} +\widehat{\phi }%
_{x}E\left\{ \widetilde{x}_{t+1}\right\} +\widehat{\phi }_{q}E\left\{ 
\widetilde{q}_{t+1}\right\} ,
\end{equation*}%
where $\widehat{\phi }_{\pi },$ $\widehat{\phi }_{x}$ and $\widehat{\phi }%
_{q}$ are now \textbf{endogenous} reaction parameters (i.e., they depend on
the fundamental parameters of our model). For that reason we first study the
stability of the following ad-hoc rule\ :%
\begin{equation}
i_{t}=\phi _{\pi }E\left\{ \pi _{H,t+1}\right\} +\phi _{x}E\left\{ 
\widetilde{x}_{t+1}\right\} +\phi _{q}E\left\{ \widetilde{q}_{t+1}\right\} ,
\label{FB-ERTR}
\end{equation}%
where $\phi _{\pi },$ $\phi _{x}$ and $\phi _{q}$ are again exogenous
non-negative reaction parameters. After that, we should check if the
parameters of the optimal rule $\widehat{\phi }_{\pi },$ $\widehat{\phi }_{x}
$ and $\widehat{\phi }_{q}$ belong to the set of parameters $\left\{ \phi
_{\pi },\phi _{x},\phi _{q}\right\} $ that report determinacy in the previous%
\textit{\ Forecast-Based Exchange Rate Taylor Rule (FB-ERTR)}.
\end{enumerate}

We should consider two things respect these simple Taylor policies. First,
under the ad-hoc simple Taylor Rules we will use the same calibration for
the policy reaction parameters as in Llosa and Tuesta (2008) when numerical
analysis is required: $\phi _{\pi }\in \left[ 0,4\right] $, $\phi _{x}\in %
\left[ 0,4\right] $ and $\phi _{s}\geq 0.$ Second, we must to decide whether
or not we consider also the forecast-based versions of these Taylor rules as
in Bullard and Mitra (2002) and Llosa and Tuesta (2008). This seems a
natural exercise because the optimal policy at the end implies an optimal
interest rate that depends on the expected values of the variables (see
below).

We compare all the results with the complete market case by setting $\kappa
_{2}=0$ and analyzing the system formed by (\ref{Phillips}) and \ref{is
equation}.

\section{Domestic Inflation Taylor Rule}

By combining the policy rule (\ref{DITR}) with (\ref{Phillips}), (\ref{is
equation}) and (\ref{exchange dynamics}), we obtain after some tedious
algebra the following representation of the competitive equilibrium under
the DITR monetary policy:\footnote{\textbf{A caveat}.- The stochatic part of
this representation should be revised: the matrix $C.$ For instance, the
definition of $\varepsilon _{t}$ includes $u_{t}.$ However, this is not
relevant for the stability of the competitive equilibrium, only for the
simulation of the dynamics and the welfare cost of deviating from the
optimal policy rule. In any case, this caveat applies for all the policy
rules.}%
\begin{equation}
\left( 
\begin{array}{c}
\mathbb{E}_{t}\left\{ \widetilde{x}_{t+1}\right\} \\ 
\mathbb{E}_{t}\left\{ \pi _{H,t+1}\right\} \\ 
\mathbb{E}_{t}\left\{ \widetilde{q}_{t+1}\right\}%
\end{array}%
\right) =\underset{A}{\underbrace{\left( 
\begin{array}{ccc}
a_{11} & a_{12} & a_{13} \\ 
a_{21} & a_{22} & a_{23} \\ 
a_{31} & a_{32} & a_{33}%
\end{array}%
\right) }}\left( 
\begin{array}{c}
\widetilde{x}_{t} \\ 
\pi _{H,t} \\ 
\widetilde{q}_{t}%
\end{array}%
\right) +\underset{C}{\underbrace{\left( 
\begin{array}{ccc}
-\frac{1}{\varpi } & 0 & \frac{\chi }{\varpi } \\ 
0 & 0 & 0 \\ 
0 & 0 & -1%
\end{array}%
\right) }}\left( 
\begin{array}{c}
\varepsilon _{t} \\ 
0 \\ 
u_{t}%
\end{array}%
\right) ,  \label{DITR solution}
\end{equation}%
where%
\begin{equation*}
a_{11}=\frac{1+\left( \phi _{x}+\frac{\lambda \kappa _{1}}{\beta }\right) %
\left[ \mu -\chi \left( 1-\gamma \right) \right] }{\varpi },
\end{equation*}%
\begin{equation*}
a_{12}=\frac{\left( \phi _{\pi }-\frac{1}{\beta }\right) \left[ \mu -\chi
\left( 1-\gamma \right) \right] }{\varpi },
\end{equation*}%
\begin{equation*}
a_{13}=\frac{\left( \frac{\lambda \kappa _{2}}{\beta }\right) \left[ \mu
-\chi \left( 1-\gamma \right) \right] -\chi }{\varpi },
\end{equation*}%
\begin{equation*}
a_{21}=-\frac{\lambda \kappa _{1}}{\beta },
\end{equation*}%
\begin{equation*}
a_{22}=\frac{1}{\beta },
\end{equation*}%
\begin{equation*}
a_{23}=-\frac{\lambda \kappa _{2}}{\beta },
\end{equation*}%
\begin{equation*}
a_{31}=\left( 1-\gamma \right) \left[ \phi _{x}-a_{21}\right] ,
\end{equation*}%
\begin{equation*}
a_{32}=\left( 1-\gamma \right) \left[ \phi _{\pi }-a_{22}\right] ,
\end{equation*}%
and%
\begin{equation*}
a_{33}=1-\left( 1-\gamma \right) a_{23}.
\end{equation*}%
For stability we must characterize the sign of the eigenvalues of matrix $A.$
Following Blanchard and Khan (1980), in our forward-looking solution (\ref%
{DITR solution}) there are three non-predeterminated variables. Therefore,
the equilibrium under DITR will be determinate if the three eigenvalues of $%
A $ are outside the unit circle, whereas it will be indeterminate when at
least one of the three eigenvalues of $A$ is inside the unit circle.
Unfortunately, there is difficult to obtain an analytical characterization
of this stability. We next simulate this stability.

\subsection{Numerical results}

Following Bullard and Mitra (2002) and Llosa and Tuesta (2008) we consider
the following policy reaction parameters: $\phi _{\pi }\in \left[ 0,4\right] 
$ and $\phi _{x}\in \left[ 0,4\right] .$ In this case we obtain the first
preliminary results:

\begin{enumerate}
\item In our baseline open economy with incomplete markets, the set of
indeterminacy is similar to the obtained by Llosa and Tuesta (2008) for the
complete market economy. See the next figure%
\begin{equation*}
\FRAME{itbpF}{4.7435in}{3.0199in}{0in}{}{}{Figure}{\special{language
"Scientific Word";type "GRAPHIC";maintain-aspect-ratio TRUE;display
"USEDEF";valid_file "T";width 4.7435in;height 3.0199in;depth
0in;original-width 11.7917in;original-height 7.4789in;cropleft "0";croptop
"1";cropright "1";cropbottom "0";tempfilename
'L37YPR02.wmf';tempfile-properties "XPR";}}
\end{equation*}%
We observe that the larger value $\phi _{\pi }$ for which indeterminacy
arises is $1,$ which corresponds with $\phi _{x}=0.$ By the contrary the
larger value of $\phi _{x}$ for which we find indeterminacy is $4$ (the
maximum value), which corresponds with $\phi _{\pi }=0.96.$ In fact this
point $\left( \phi _{\pi },\phi _{x}\right) =\left( 1,0.96\right) ,$ which
we mark with $H$ in the figure, determines the length of indeterminacy and,
therefore, the constraint faced by the policy makers in setting a DITR as a
stabilizing device. In particular, the monetary authority has not constraint
if the policy reaction to inflation $\phi _{\pi }$ is larger than unity.
However, provided that $\phi _{\pi }<1,$ the smaller this policy parameter
is, the greater the authority's responses to the output gap.

TASK.- We should check the sensitivity of the indeterminacy length (point H)
to the degree of openness ($\gamma $ and $\delta )$ like in Llosa and Tuesta
(2008).

\item \textbf{Main result.- }The set of indeterminacy in our version of open
economy with complete markets and close economy are similar to the previous
one. However, the indeterminacy is larger in the incomplete market economy
than in the complete market one. In addition, the length of indeterminacy in
the close economy is between the length in the two open economies: with and
without complete markets. The largest value of $\phi _{\pi }$ for which
indeterminacy arises is $1$ in the three economies. However the point $H$ in
the previous figure is different across economies. Next table provides the
value of this point for the three economies:%
\begin{equation*}
\begin{tabular}{lcccc}
\multicolumn{5}{c}{Indeterminacy threshold (point $H)$} \\ \hline\hline
&  & $\phi _{\pi }$ &  & $\phi _{x}$ \\ \hline
\multicolumn{1}{c}{Open and incomplete} &  & $0.96$ &  & $4$ \\ 
\multicolumn{1}{c}{Open and complete} &  & $0.65$ &  & $4$ \\ 
\multicolumn{1}{c}{Close} &  & $0.91$ &  & $4$ \\ \hline
\end{tabular}%
\end{equation*}%
Therefore, the constrains faced by the policy markers is slightly tighter in
the baseline economy (i.e., open economy with incomplete markets) than in
the open economy with complete markets. This means that, whenever $\phi
_{\pi }<1,$ open economies with incomplete markets needs greater responses
to the output gap than open economies with complete markets. The mechanism
should be found in the parameters $\kappa _{2}$ and $\kappa _{1}$
determining the response of inflation rate to the current exchange rate and
the current output gap, respectively. WE MUST WORK THIS POINT. Furthermore,
as in Llosa and Tuesta (2008), the constrains faced by the policy markers is
slightly tighter in the close economy than in the open economy with complete
markets. However, and this is a remarkable result, the constraints faced by
the policy makers is slightly tighter in the open economy with incomplete
markets than in the close economy. In other words, while openness reduces
the constraint for policy makers if markets are complete, this opens
increases the constraints if markets are incomplete.

\item As was mentioned before, the results largely depend on the value of
sigma. First, the length of indeterminacy is increasing in $\sigma .$ In
addition, the quantitative differences in the set of indeterminacy across
economies is slightly larger for smaller values of $\sigma .$ The next table
reports the results on the length of indeterminacy for $\sigma =1$
(considered by Gali and Monacelli, 2005):%
\begin{equation*}
\begin{tabular}{lcccc}
\multicolumn{5}{c}{Indeterminacy threshold (point $H)$} \\ \hline\hline
&  & $\phi _{\pi }$ &  & $\phi _{x}$ \\ \hline
\multicolumn{1}{c}{Open and incomplete} &  & $0.82$ &  & $4$ \\ 
\multicolumn{1}{c}{Open and complete} &  & $0.59$ &  & $4$ \\ 
\multicolumn{1}{c}{Close} &  & $0.68$ &  & $4$ \\ \hline
\end{tabular}%
\end{equation*}

\item The value of $\phi $ (endogenous discounting) has either a marginal
role in the differences across the economies. However, by setting $\phi =0,$
the competitive equilibrium is almost indeterminate for all the combinations
of the policy reaction parameters $\phi _{\pi }$ and $\phi _{x}.$ In any
case, it is difficult to find any pattern in the indeterminacy when $\phi
=0. $
\end{enumerate}

\textbf{In summarizing}, indeterminacy under this Domestic Inflation Taylor
Rule in our model is more often than in the original Gali and Monacelli
(2005), which was analyzed by Bullard and Mitra (2002) and Llosa and Tuesta
(2008). Furthermore, contrary to those authors, openness with incomplete
markets increases the combinations of parameter reactions $\phi _{\pi }$ and 
$\phi _{x}$ for which equilibrium in indeterminate.

\subsection{Forecast-based DITR}

We now analyze the stability with the monetary authority sets the domestic
interest rate by responding to the expected domestic inflation and the
domestic output gap:%
\begin{equation}
i_{t}=\phi _{\pi }\mathbb{E}_{t}\left\{ \pi _{H,t+1}\right\} +\phi _{x}%
\mathbb{E}_{t}\left\{ \widetilde{x}_{t+1}\right\} .  \label{EDITR}
\end{equation}%
With this policy rule, the competitive equilibrium is given by%
\begin{equation}
\left( 
\begin{array}{c}
\mathbb{E}_{t}\left\{ \widetilde{x}_{t+1}\right\} \\ 
\mathbb{E}_{t}\left\{ \pi _{H,t+1}\right\} \\ 
\mathbb{E}_{t}\left\{ \widetilde{q}_{t+1}\right\}%
\end{array}%
\right) =\underset{A_{1}}{\underbrace{\left( 
\begin{array}{ccc}
a_{11} & a_{12} & a_{13} \\ 
a_{21} & a_{22} & a_{23} \\ 
a_{31} & a_{32} & a_{33}%
\end{array}%
\right) }}\left( 
\begin{array}{c}
\widetilde{x}_{t} \\ 
\pi _{H,t} \\ 
\widetilde{q}_{t}%
\end{array}%
\right) +\underset{C_{1}}{\underbrace{\left( 
\begin{array}{ccc}
c_{11} & 0 & c_{13} \\ 
0 & 0 & 0 \\ 
c_{31} & 0 & c_{33}%
\end{array}%
\right) }}\left( 
\begin{array}{c}
\varepsilon _{t} \\ 
0 \\ 
u_{t}%
\end{array}%
\right) ,  \label{EDITR solution}
\end{equation}%
where 
\begin{equation*}
a_{11}=\frac{1+\left( \frac{\lambda \kappa _{1}}{\beta }\right) \left(
1-\phi _{\pi }\right) \left[ \mu -\chi \left( 1-\gamma \right) \right] }{%
\varpi -\phi _{x}\left[ \mu -\chi \left( 1-\gamma \right) \right] },
\end{equation*}%
\begin{equation*}
a_{12}=-\frac{\left( \frac{1}{\beta }\right) \left( 1-\phi _{\pi }\right) %
\left[ \mu -\chi \left( 1-\gamma \right) \right] }{\varpi -\phi _{x}\left[
\mu -\chi \left( 1-\gamma \right) \right] },
\end{equation*}%
\begin{equation*}
a_{13}=\frac{-\chi +\left( \frac{\lambda \kappa _{2}}{\beta }\right) \left(
1-\phi _{\pi }\right) \left[ \mu -\chi \left( 1-\gamma \right) \right] }{%
\varpi -\phi _{x}\left[ \mu -\chi \left( 1-\gamma \right) \right] },
\end{equation*}%
\begin{equation*}
a_{21}=-\frac{\lambda \kappa _{1}}{\beta },
\end{equation*}%
\begin{equation*}
a_{22}=\frac{1}{\beta },
\end{equation*}%
\begin{equation*}
a_{23}=-\frac{\lambda \kappa _{2}}{\beta },
\end{equation*}%
\begin{equation*}
a_{31}=\left( 1-\gamma \right) \left[ a_{11}\phi _{x}-a_{21}\left( 1-\phi
_{\pi }\right) \right] ,
\end{equation*}%
\begin{equation*}
a_{32}=\left( 1-\gamma \right) \left[ a_{12}\phi _{x}-a_{22}\left( 1-\phi
_{\pi }\right) \right] ,
\end{equation*}%
\begin{equation*}
a_{33}=1+\left( 1-\gamma \right) \left[ a_{13}\phi _{x}-a_{23}\left( 1-\phi
_{\pi }\right) \right] ,
\end{equation*}%
\begin{equation*}
c_{11}=-\frac{1}{\varpi -\phi _{x}\left[ \mu -\chi \left( 1-\gamma \right) %
\right] },
\end{equation*}%
\begin{equation*}
c_{13}=\frac{\chi }{\varpi -\phi _{x}\left[ \mu -\chi \left( 1-\gamma
\right) \right] },
\end{equation*}%
\begin{equation*}
c_{31}=-\frac{\left( 1-\gamma \right) \left( 1-\phi _{\pi }\right) }{\varpi
-\phi _{x}\left[ \mu -\chi \left( 1-\gamma \right) \right] },
\end{equation*}%
and%
\begin{equation*}
c_{33}=\frac{\chi \left( 1-\gamma \right) \left( 1-\phi _{\pi }\right) }{%
\varpi -\phi _{x}\left[ \mu -\chi \left( 1-\gamma \right) \right] }-1.
\end{equation*}%
The results of stability in this case are (\textbf{CHECK ALGEBRA}):

\begin{enumerate}
\item There are a subset of combinations of $\phi _{\pi }$ and $\phi _{x}$
where the competitive equilibrium is indeterminate. Now this subset of
indeterminacy is much larger than in the case of DITR.

\item Indeterminacy is larger in incomplete than in complete.

\item Indeterminacy is larger in the open economy with complete markets than
in close economy. This result contradicts Llosa and Tuesta (2008). MORE WORK
IS REQUIRED.

\item BUT, the picture are quite different that those in Llosa and Tuesta
(2008). What is wrong? MORE WORK IS REQUIRED.
\end{enumerate}

These results can be obtained from the following figure%
\begin{equation*}
\begin{tabular}{lll}
$\FRAME{itbpF}{1.7694in}{1.1355in}{0in}{}{}{Figure}{\special{language
"Scientific Word";type "GRAPHIC";maintain-aspect-ratio TRUE;display
"USEDEF";valid_file "T";width 1.7694in;height 1.1355in;depth
0in;original-width 11.6144in;original-height 7.3855in;cropleft "0";croptop
"1";cropright "1";cropbottom "0";tempfilename
'L3823R07.wmf';tempfile-properties "XPR";}}$ & \FRAME{itbpF}{1.7746in}{%
1.1684in}{0in}{}{}{Figure}{\special{language "Scientific Word";type
"GRAPHIC";maintain-aspect-ratio TRUE;display "USEDEF";valid_file "T";width
1.7746in;height 1.1684in;depth 0in;original-width 11.6456in;original-height
7.6043in;cropleft "0";croptop "1";cropright "1";cropbottom "0";tempfilename
'L381S105.wmf';tempfile-properties "XPR";}} & \FRAME{itbpF}{1.7288in}{%
1.1467in}{0in}{}{}{Figure}{\special{language "Scientific Word";type
"GRAPHIC";maintain-aspect-ratio TRUE;display "USEDEF";valid_file "T";width
1.7288in;height 1.1467in;depth 0in;original-width 11.3438in;original-height
7.4581in;cropleft "0";croptop "1";cropright "1";cropbottom "0";tempfilename
'L381Z806.wmf';tempfile-properties "XPR";}}%
\end{tabular}%
\end{equation*}

\section{CPI Inflation Taylor Rule}

The problem with this rule is that our reduced form of equilibrium is in
terms of the domestic inflation $\pi _{H,t}$ instead of CPI inflation $\pi
_{t}.$ However, we can find in the solution of the model a relations between
those two variables that we can use to analyze the stability of this rule.
Effectively, we know that%
\begin{equation}
\pi _{t}=\pi _{H,t}+\left( \frac{\gamma }{1-\gamma }\right) \left(
q_{t}-q_{t-1}\right) .  \label{CPI}
\end{equation}%
By using (\ref{CPI}) into  the policy rule (\ref{CPITR}), we obtain%
\begin{equation}
i_{t}=\phi _{\pi }\pi _{H,t}+\phi _{x}\widetilde{x}_{t}+\left( \frac{\gamma
\phi _{\pi }}{1-\gamma }\right) \left( \widetilde{q}_{t}-\widetilde{q}%
_{t-1}\right) .  \label{DICPITR}
\end{equation}

By combining the policy rule (\ref{DICPITR}) with (\ref{Phillips}), (\ref{is
equation}) and (\ref{exchange dynamics}), we obtain after some tedious
algebra the representation of the competitive equilibrium under the CPITR
monetary policy. We omit the representation (TO\ BE\ COMPLETED).

\subsection{Results}

For stability we introduce a dummy variable $z_{t}=\widetilde{q}_{t-1}.\ $%
Hence, the reduced form of the equilibrium is now expressed in terms of the
variables $\pi _{H,t},$ $\widetilde{x}_{t},$ $\widetilde{q}_{t}$ and $z_{t}.$
Since $z_{t}$ is a predeterminated variable, the equilibrium under CPITR is
determinate if three eigenvalues are outside the unit circle. If less than
three eigenvalues are outside the unit circle, then the equilibrium is
indeterminate. We next simulate this stability. Following Bullard and Mitra
(2002) and Llosa and Tuesta (2008) we consider the following policy reaction
parameters: $\phi _{\pi }\in \left[ 0,4\right] $ and $\phi _{x}\in \left[ 0,4%
\right] .$ In this case we obtain the following results:

\begin{enumerate}
\item Under incomplete markets, the shape of the set of indeterminacy is
similar to the obtained by Llosa and Tuesta (2008) for the complete market
economy$.$ As those authors, this set is equal to the obtained under the
DITR rule. Therefore, the constraint faced by the policy makers in setting a
CPITR rule are identical to the one in setting a DITR rule. See the figure:%
\begin{equation*}
\FRAME{itbpF}{5.0246in}{3.2344in}{0in}{}{}{Figure}{\special{language
"Scientific Word";type "GRAPHIC";maintain-aspect-ratio TRUE;display
"USEDEF";valid_file "T";width 5.0246in;height 3.2344in;depth
0in;original-width 11.1042in;original-height 7.1252in;cropleft "0";croptop
"1";cropright "1";cropbottom "0";tempfilename
'L5F7A400.wmf';tempfile-properties "XPR";}}
\end{equation*}%
We observe that the larger value of $\phi _{\pi }$ for which indeterminacy
arises is $1,$ which corresponds with $\phi _{x}=0.$ By the contrary the
larger value of $\phi _{x}$ for which we find indeterminacy is $4$ (the
maximum value), which corresponds with $\phi _{\pi }=0.96.$ In particular,
provided that $\phi _{\pi }<1,$ the smaller this policy parameter is, the
greater the authority's responses to the output gap. This coincides with the
result obtained under the DITR rule.

\item Under complete markets we use the fact that $c_{t}=q_{t}/\sigma .$
This implies that $q_{t}=\tau y_{t},$ where 
\begin{equation*}
\tau =\frac{\sigma \left( 1-\gamma \right) }{\left( 1-\gamma \right)
^{2}+\sigma \eta \gamma \left( 2-\gamma \right) }.
\end{equation*}%
Thus, the policy rule can be transformed into%
\begin{equation*}
i_{t}=\phi _{\pi }\pi _{H,t}+\left[ \phi _{x}+\frac{\tau \gamma \phi _{\pi }%
}{1-\gamma }\right] \widetilde{x}_{t}-\left( \frac{\tau \gamma \phi _{\pi }}{%
1-\gamma }\right) \widetilde{x}_{t-1},
\end{equation*}%
and we now use the dummy variable $z_{t}=\widetilde{x}_{t-1}.$ With complete
markets, the results coincide with Llosa and Tuesta (2008). Furthermore, the
shape of the indeterminacy set is identical to the one obtained in the
economy with incomplete markets. However, the indeterminacy is larger in the
incomplete market economy than in the complete market one. Next table shows
how different is the set of indeterminacy among the two economies by
providing the coordinates of $\phi _{\pi }$ and $\phi _{x}$ that defines
this set.%
\begin{equation*}
\begin{tabular}{lcccccccc}
\multicolumn{9}{c}{Length of Indeterminacy with CPITR} \\ \hline\hline
& \multicolumn{8}{c}{Vertices of the indeterminacy set} \\ \hline
\multicolumn{1}{c}{Open and incomplete} &  & $\left( 0,0\right) $ &  & $%
\left( 1,0\right) $ &  & $\left( 0.96,4\right) $ &  & $\left( 0,4\right) $
\\ 
\multicolumn{1}{c}{Open and complete} &  & $\left( 0,0\right) $ &  & $\left(
1,0\right) $ &  & $\left( 0.65,4\right) $ &  & $\left( 0,4\right) $ \\ 
\multicolumn{1}{c}{Close} &  & $\left( 0,0\right) $ &  & $\left( 1,0\right) $
&  & $\left( 0.91,4\right) $ &  & $\left( 0,4\right) $ \\ \hline
\end{tabular}%
\end{equation*}

\item The previous table also shows the length of indeterminacy in the case
of the close economy. The numerical result is exactly the same as the one
with the DITR\ rule. More important, as in the case of DITR, the
indeterminacy in the close economy is larger than in the case of the open
economy with complete markets (result also in Llosa and Tuesta, 2008),
whereas is a little smaller than in the case of the open economy with
incomplete markets.

\item \textbf{Main conclusion.}- The stability results in CPITR and DITR are
identical. Hence, it does not matter whether the monetary authority cares
about domestic inflation or CPI inflation when she does not care about
exchange rate. We next show that this is not true when the authority follows
a forecast-based rule.
\end{enumerate}

\subsection{Forecast-based CPI Inflation Taylor Rule}

We now analyze the stability with the monetary authority sets the domestic
interest rate by responding to the expected CPI inflation and the expected
domestic output gap :%
\begin{equation*}
i_{t}=\phi _{\pi }\mathbb{E}_{t}\left\{ \pi _{t+1}\right\} +\phi _{x}\mathbb{%
E}_{t}\left\{ \widetilde{x}_{t+1}\right\} 
\end{equation*}%
By following the same procedure than in the case of the MERTR rule, we
obtain that this rate is equivalent to%
\begin{equation*}
i_{t}=\phi _{\pi }\mathbb{E}_{t}\left\{ \pi _{H,t+1}\right\} +\phi _{x}%
\mathbb{E}_{t}\left\{ \widetilde{x}_{t+1}\right\} +\left( \frac{\gamma \phi
_{\pi }}{1-\gamma }\right) \left[ \mathbb{E}_{t}\left\{ \widetilde{q}%
_{t+1}\right\} -\widetilde{q}_{t}\right] 
\end{equation*}%
in the case of incomplete markets, and to%
\begin{equation*}
i_{t}=\phi _{\pi }\mathbb{E}_{t}\left\{ \pi _{H,t+1}\right\} +\left( \phi
_{x}+\frac{\tau \gamma \phi _{\pi }}{1-\gamma }\right) \mathbb{E}_{t}\left\{ 
\widetilde{x}_{t+1}\right\} -\left( \frac{\tau \gamma \phi _{\pi }}{1-\gamma 
}\right) \widetilde{x}_{t},
\end{equation*}%
in the case of complete markets.

The stability results in this case as the follow:

\begin{enumerate}
\item The indeterminacy set follows in the economy with incomplete markets
similar patterns found by LLosa and Tuesta (2008) for the complete market
economy. See the graph:%
\begin{equation*}
\FRAME{itbpF}{4.9441in}{3.1825in}{0in}{}{}{Figure}{\special{language
"Scientific Word";type "GRAPHIC";maintain-aspect-ratio TRUE;display
"USEDEF";valid_file "T";width 4.9441in;height 3.1825in;depth
0in;original-width 10.9269in;original-height 7.0102in;cropleft "0";croptop
"1";cropright "1";cropbottom "0";tempfilename
'L5F9FV01.wmf';tempfile-properties "XPR";}}
\end{equation*}

\item Again the indeterminacy set is larger in the economy with incomplete
markets than in the case of complete markets economy. In fact, the
combinations of the reaction parameters $\phi _{\pi }$ and $\phi _{x}$ for
which the equilibrium under FB-CPITR is determinate is very small. 

\item The indeterminacy set is much smaller in the case of the close economy
than in the open economy (with and without incomplete markets). However, the
set of determinacy is quite different in our case. In our case, this set is
identical to the one obtained for the case of close economy with FB-DITR.

\item Indeterminacy is much larger under FB-CPITR than in the case of
FB-DITR. Therefore, the monetary authority faces to larger constraints in
fixing the FB-CPITR than the FB-DITR, that is, when she cares about the CPI
inflation than the Domestic inflation.
\end{enumerate}

\section{Managed Exchange Rate Taylor Rule}

The problem with this rule is that our reduced form of equilibrium is in
terms of $\pi _{H,t}$ and $\widetilde{q}_{t}$ instead of $\pi _{t}$ and $%
\bigtriangleup s_{t}.$ However, we can find in the solution of the model a
relations between those two variables that we can use to analyze the
stability of this rule. On the one hand, we first use the relation (\ref{CPI}%
). On the other hand, by definition we know that%
\begin{equation*}
s_{t}=q_{t}+p_{t}-p_{t}^{\ast }.
\end{equation*}%
From this expression we directly obtain that%
\begin{equation}
\bigtriangleup s_{t}=\bigtriangleup q_{t}+\bigtriangleup
p_{t}-\bigtriangleup p_{t}^{\ast }.  \label{Variation of s}
\end{equation}%
We derived in the paper (see Expression 28) that%
\begin{equation*}
p_{t}-p_{H,t}=\left( \frac{\gamma }{1-\gamma }\right) q_{t}.
\end{equation*}%
From this expression we also directly obtain that%
\begin{equation*}
\bigtriangleup p_{t}=\left( \frac{\gamma }{1-\gamma }\right) \bigtriangleup
q_{t}+\bigtriangleup p_{H,t},
\end{equation*}%
which can be written by using the definition of $\pi _{H,t}$ as%
\begin{equation}
\bigtriangleup p_{t}=\left( \frac{\gamma }{1-\gamma }\right) \bigtriangleup
q_{t}+\pi _{H,t}.  \label{Variation of p}
\end{equation}%
By combining (\ref{Variation of s}) and (\ref{Variation of p}), we get%
\begin{equation}
\bigtriangleup s_{t}=\left( \frac{1}{1-\gamma }\right) \bigtriangleup
q_{t}+\pi _{H,t}-\bigtriangleup p_{t}^{\ast }.  \label{Variation of s bis}
\end{equation}%
At this point, we assume that the rest of the world follows an optimal
monetary policy such that $\bigtriangleup p_{t}^{\ast }=0.$ This is without
lost of generality since $\bigtriangleup p_{t}^{\ast }$ is an exogenous
process in general, so it does not affect stability of equilibria. We then
obtain the following equilibrium condition%
\begin{equation}
\bigtriangleup s_{t}=\left( \frac{1}{1-\gamma }\right) \bigtriangleup
q_{t}+\pi _{H,t}.  \label{our relation}
\end{equation}

By using (\ref{CPI}) and (\ref{our relation}) in the policy rule (\ref{MERTR}%
), we obtain%
\begin{equation}
i_{t}=\left( \phi _{\pi }+\phi _{s}\right) \pi _{H,t}+\phi _{x}\widetilde{x}%
_{t}+\left( \frac{\gamma \phi _{\pi }+\phi _{s}}{1-\gamma }\right) \left( 
\widetilde{q}_{t}-\widetilde{q}_{t-1}\right) .  \label{DIMERTR}
\end{equation}

By combining the policy rule (\ref{DIMERTR}) with (\ref{Phillips}), (\ref{is
equation}) and (\ref{exchange dynamics}), we obtain after some tedious
algebra the representation of the competitive equilibrium under the MERTR
monetary policy. We omit the representation (TO\ BE\ COMPLETED).

\subsection{Results}

For stability we introduce a dummy variable $z_{t}=\widetilde{q}_{t-1}.\ $%
Hence, the reduced form of the equilibrium is now expressed in terms of the
variables $\pi _{H,t},$ $\widetilde{x}_{t},$ $\widetilde{q}_{t}$ and $z_{t}.$
Since $z_{t}$ is a predeterminated variable, the equilibrium under MERTR is
determinate if three eigenvalues are outside the unit circle. If less than
three eigenvalues are outside the unit circle, then the equilibrium is
indeterminate. We next simulate this stability. Following Bullard and Mitra
(2002) and Llosa and Tuesta (2008) we consider the following policy reaction
parameters: $\phi _{\pi }\in \left[ 0,4\right] ,$ $\phi _{x}\in \left[ 0,4%
\right] $ and $\phi _{s}=0.6$. WE SHOULD MAKE SENSITIVITY ANALYSIS RESPECT
TO $\phi _{s}.$ In this case we obtain the following results:

\begin{enumerate}
\item Under incomplete markets, the shape of the set of indeterminacy is
similar to the obtained by Llosa and Tuesta (2008) for the complete market
economy$.$ As those authors, this set is similar to the obtained under the
DITR rule. However, the set of indeterminacy is also in our case smaller
under the MERTR rule than under the DITR. Therefore, the MERTR rule is a
more effective stabilizing policy, i.e., the constraint faced by the policy
makers in setting a MERTR rule is smaller than the one in setting a DITR
rule. See the figure%
\begin{equation*}
\FRAME{itbpF}{4.5316in}{2.8279in}{0in}{}{}{Figure}{\special{language
"Scientific Word";type "GRAPHIC";maintain-aspect-ratio TRUE;display
"USEDEF";valid_file "T";width 4.5316in;height 2.8279in;depth
0in;original-width 11.2607in;original-height 6.9998in;cropleft "0";croptop
"1";cropright "1";cropbottom "0";tempfilename
'L59YT900.wmf';tempfile-properties "XPR";}}
\end{equation*}

We observe that the larger value of $\phi _{\pi }$ for which indeterminacy
arises is $0.39,$ which corresponds with $\phi _{x}=0.$ By the contrary the
larger value of $\phi _{x}$ for which we find indeterminacy is $4$ (the
maximum value), which corresponds with $\phi _{\pi }=0.36.$ In particular,
provided that $\phi _{\pi }<0.39,$ the smaller this policy parameter is, the
greater the authority's responses to the output gap.

\item Under complete markets we use the fact that $c_{t}=q_{t}/\sigma .$
This implies that $q_{t}=\tau y_{t},$ where 
\begin{equation*}
\tau =\frac{\sigma \left( 1-\gamma \right) }{\left( 1-\gamma \right)
^{2}+\sigma \eta \gamma \left( 2-\gamma \right) }.
\end{equation*}%
Thus, the policy rule can be transformed into%
\begin{equation*}
i_{t}=\left( \phi _{\pi }+\phi _{s}\right) \pi _{H,t}+\left[ \phi _{x}+\frac{%
\tau \left( \gamma \phi _{\pi }+\phi _{s}\right) }{1-\gamma }\right] 
\widetilde{x}_{t}-\left( \frac{\tau \left( \gamma \phi _{\pi }+\phi
_{s}\right) }{1-\gamma }\right) \widetilde{x}_{t-1},
\end{equation*}%
and we now use the dummy variable $z_{t}=\widetilde{x}_{t-1}.$ With complete
markets, the results coincide with Llosa and Tuesta (2008). Furthermore, the
shape of the indeterminacy set is identical to the one obtained in the
economy with incomplete markets. However, the indeterminacy is larger in the
incomplete market economy than in the complete market one. Next table shows
how different is the set of indeterminacy among the two economies by
providing the coordinates of $\phi _{\pi }$ and $\phi _{x}$ that defines
this set.%
\begin{equation*}
\begin{tabular}{lcccccccc}
\multicolumn{9}{c}{Length of Indeterminacy with MERTR} \\ \hline\hline
& \multicolumn{8}{c}{Vertices of the indeterminacy set} \\ \hline
\multicolumn{1}{c}{Open and incomplete} &  & $\left( 0,0\right) $ &  & $%
\left( 0.39,0\right) $ &  & $\left( 0.35,4\right) $ &  & $\left( 0,4\right) $
\\ 
\multicolumn{1}{c}{Open and complete} &  & $\left( 0,0\right) $ &  & $\left(
0.39,0\right) $ &  & $\left( 0.04,4\right) $ &  & $\left( 0,4\right) $ \\ 
\hline
\end{tabular}%
\end{equation*}

\item \textbf{Main conclusions}.- The MERTR is a more effective policy to
stabilize the economy than the DITR. In other words, the equilibrium under
DITR is indeterminate for a larger number of combinations of the reaction
parameters $\phi _{\pi }$ and $\phi _{x}$ than the equilibrium under MERTR,
provided that the authority reacts to changes in the nominal exchange rate,
i.e., $\phi _{s}>0.$ Under both rules the indeterminacy set is always larger
in the economy with incomplete markets than in the economy with complete
markets. Hence, the constraints faced for the monetary authority in
stabilizing the economy is larger in the economy with incomplete markets.
\end{enumerate}

\subsection{Forecast-based managed exchange rate Taylor rule}

We now analyze the stability with the monetary authority sets the domestic
interest rate by responding to the expected CPI inflation, the expected
domestic output gap and the expected variation on the nominal exchange rate:%
\begin{equation*}
i_{t}=\phi _{\pi }\mathbb{E}_{t}\left\{ \pi _{t+1}\right\} +\phi _{x}\mathbb{%
E}_{t}\left\{ \widetilde{x}_{t+1}\right\} +\phi _{s}\mathbb{E}_{t}\left\{
\bigtriangleup s_{t+1}\right\} .
\end{equation*}%
By following the same procedure than in the case of the MERTR rule, we
obtain that this rate is equivalent to%
\begin{equation*}
i_{t}=\left( \phi _{\pi }+\phi _{s}\right) \mathbb{E}_{t}\left\{ \pi
_{H,t+1}\right\} +\phi _{x}\mathbb{E}_{t}\left\{ \widetilde{x}_{t+1}\right\}
+\left( \frac{\gamma \phi _{\pi }+\phi _{s}}{1-\gamma }\right) \left[ 
\mathbb{E}_{t}\left\{ \widetilde{q}_{t+1}\right\} -\widetilde{q}_{t}\right]
\end{equation*}%
in the case of incomplete markets, and to%
\begin{equation*}
i_{t}=\left( \phi _{\pi }+\phi _{s}\right) \mathbb{E}_{t}\left\{ \pi
_{H,t+1}\right\} +\left[ \phi _{x}+\tau \left( \frac{\gamma \phi _{\pi
}+\phi _{s}}{1-\gamma }\right) \right] \mathbb{E}_{t}\left\{ \widetilde{x}%
_{t+1}\right\} -\tau \left( \frac{\gamma \phi _{\pi }+\phi _{s}}{1-\gamma }%
\right) \widetilde{x}_{t},
\end{equation*}%
in the case of complete markets.

The stability results in this case as the follow:

\begin{enumerate}
\item The indeterminacy set follows in the economy with incomplete markets
the same patterns found by LLosa and Tuesta (2008) for the complete market
economy. See the graph:%
\begin{equation*}
\FRAME{itbpF}{3.8052in}{3.1012in}{0in}{}{}{Figure}{\special{language
"Scientific Word";type "GRAPHIC";maintain-aspect-ratio TRUE;display
"USEDEF";valid_file "T";width 3.8052in;height 3.1012in;depth
0in;original-width 6.9894in;original-height 5.6879in;cropleft "0";croptop
"1";cropright "1";cropbottom "0";tempfilename
'L5CL4V00.wmf';tempfile-properties "XPR";}}
\end{equation*}

\item Again the indeterminacy set is larger in the economy with incomplete
markets than in the case of complete markets economy. In fact, the
combinations of the reaction parameters $\phi _{\pi }$ and $\phi _{x}$ for
which the equilibrium under FB-MERTR is determinate is very small. I

\item In this case, the indeterminacy set is larger than in the case of the
FB-DITR rule. Remember that the opposite result arises by comparing MERTR\
and DITR. In this case the indeterminacy is larger in the case of DITR.
\end{enumerate}

\section{Optimal rule}

TO BE COMPLETED

The problem is that this optimal rule was derived with $\phi =0,$ and the
indeterminacy follows a strange pattern in this case and arises for almost
the entire set of combinations of $\phi _{\pi }$ and $\phi _{x}.$ In any
case, we can still find the endogenous values of the policy reaction
parameters $\phi _{\pi ,}\phi _{x}$ and $\phi _{q},$ and then see whether
this combination belongs to the determinacy or indeterminacy region found
for the Forecast-based Exchange Rate Taylor Rule. This FB-ERTR policy is of
the same form as the optimal policy, with the difference that in the former
rule the policy reaction parameters are exogenous. An important \textbf{%
caveat} of this procedure is that the stability of the FB-ERTR was derived
with $\phi >0,$ while the optimal policy was derived with $\phi =0.$
However, we can obvious this problem because in the baseline calibration the
value of $\phi $ is too close zero.

IS THIS STRATEGY CORRECT?

\subsection{Forecast-Based Exchange Rate Taylor Rule}

Hence, we first analyze the case of the Forecast-Based Exchange Rate Taylor
Rule. By combining the policy rule (\ref{FB-ERTR}) with (\ref{Phillips}), (%
\ref{is equation}) and (\ref{exchange dynamics}), we obtain after some
tedious algebra the representation of the competitive equilibrium under the
FB-ERTR monetary policy (TO BE COMPLETED).

We next simulate this stability. Following Bullard and Mitra (2002) and
Llosa and Tuesta (2008) we consider the following policy reaction
parameters: $\phi _{\pi }\in \left[ 0,4\right] $ and $\phi _{x}\in \left[ 0,4%
\right] .$ For the value of $\phi _{q}$ we take the value of its counterpart
in the optimal policy $\widehat{\phi }_{q}.$ (CHECK THAT $\widehat{\phi }%
_{\pi }\in \left[ 0,4\right] $ and $\widehat{\phi }_{x}\in \left[ 0,4\right]
.$ IF NOT ENLARGE THE VALUES OF $\phi _{\pi }$ AND $\phi _{x}$ IN THE
SIMULATIONS). In this case we obtain the first preliminary results:

\begin{enumerate}
\item TO BE COMPLETE
\end{enumerate}

\subsection{Stability of the optimal rule}

TO BE COMPLETED

\end{document}
